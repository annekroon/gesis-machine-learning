\section{Regular expressions}

\subsection{What is a regexp?}
\begin{frame}{Regular Expressions: What and why?}
\begin{block}{What is a regexp?}
\begin{itemize}
\item<1-> a \emph{very} widespread way to describe patterns in strings
\item<2-> Think of wildcards like {\tt{*}} or operators like {\tt{OR}}, {\tt{AND}} or {\tt{NOT}} in search strings: a regexp does the same, but is \emph{much} more powerful
\item<3-> You can use them in many editors (!), in the Terminal, in STATA \ldots and in Python
\end{itemize}
\end{block}
\end{frame}

\begin{frame}{A more powerful tool}
\begin{block}{An example}
\begin{itemize}
\item We want to remove everything but words from a tweet
\item We can do so by calling the \texttt{.replace()} method multiple times (for each unwanted character)
\item But we can better do this with a regular expression instead: \\
{\tt{ \lbrack \^{}a-zA-Z\rbrack}}  matches anything that is not a letter
\end{itemize}
\end{block}
\end{frame}

\begin{frame}{Basic regexp elements}
\begin{block}{Alternatives}
\begin{description}
\item[{\tt{\lbrack TtFf\rbrack}}] matches either T or t or F or f
\item[{\tt{Twitter|Facebook}}] matches either Twitter or Facebook
\item[{\tt{.}}] matches any character
\end{description}
\end{block}
\begin{block}{Repetition}<2->
\begin{description}
\item[{\tt{?}}] the expression before occurs 0 or 1 times
\item[{\tt{*}}] the expression before occurs 0 or more times
\item[{\tt{+}}] the expression before occurs 1 or more times
\end{description}
\end{block}
\end{frame}

\begin{frame}{regexp quizz}
\begin{block}{Which words would be matched?}
\tt
\begin{enumerate}
\item<1-> \lbrack Pp\rbrack ython
\item<2-> \lbrack A-Z\rbrack +
\item<3-> RT ?:? @\lbrack a-zA-Z0-9\rbrack +
\end{enumerate}
\end{block}
\end{frame}

\begin{frame}{What else is possible?}
See the table in the book!
\end{frame}

\subsection{Using a regexp in Python}
\begin{frame}{How to use regular expressions in Python}
\begin{block}{The module \texttt{re}*}
\begin{description}
\item<1->[{\tt{re.findall("\lbrack Tt\rbrack witter|\lbrack Ff\rbrack acebook",testo)}}] returns a list with all occurances of Twitter or Facebook in the string called {\tt{testo}}
\item<1->[{\tt{re.findall("\lbrack 0-9\rbrack +\lbrack a-zA-Z\rbrack +",testo)}}] returns a list with all words that start with one or more numbers followed by one or more letters in the string called {\tt{testo}}
\item<2->[{\tt{re.sub("\lbrack Tt\rbrack witter|\lbrack Ff\rbrack acebook","a social medium",testo)}}] returns a string in which all all occurances of Twitter or Facebook are replaced by "a social medium"
\end{description}
\end{block}

\tiny{Use the less-known but more powerful module \texttt{regex} instead to support all dialects used in the book}
\end{frame}


\begin{frame}[fragile]{How to use regular expressions in Python}
\begin{block}{The module re}
\begin{description}
\item<1->[{\tt{re.match(" +(\lbrack 0-9\rbrack +) of (\lbrack 0-9\rbrack +) points",line)}}] returns  \texttt{None} unless it \emph{exactly} matches the string \texttt{line}. If it does, you can access the part between () with the \texttt{.group()} method.
\end{description}
\end{block}

Example:
\begin{lstlisting}
line="             2 of 25 points"
result=re.match(" +([0-9]+) of ([0-9]+) points",line)
if result:
   print ("Your points:",result.group(1))
   print ("Maximum points:",result.group(2))
\end{lstlisting}
Your points: 2\\
Maximum points: 25
\end{frame}














\begin{frame}{Possible applications}
\begin{block}{Data preprocessing}
\begin{itemize}
\item Remove unwanted characters, words, \ldots
\item Identify \emph{meaningful} bits of text: usernames, headlines, where an article starts, \ldots
\item filter (distinguish relevant from irrelevant cases)
\end{itemize}
\end{block}
\end{frame}


\begin{frame}{Possible applications}
\begin{block}{Data analysis: Automated coding}
\begin{itemize}
\item Actors
\item Brands
\item links or other markers that follow a regular pattern
\item Numbers (!)
\end{itemize}
\end{block}
\end{frame}

\begin{frame}[fragile,plain]{Example 1: Counting actors}
\begin{minted}{python}
import re, csv
from glob import glob
counts1=[]
counts2=[]
filenames = glob("/home/damian/articles/*.txt")

for fn in filenames:
   with open(fn) as fi:
      artikel = fi.read()
      artikel = artikel.replace('\n',' ')
      
      counts1.append(len(re.findall('Israel.*(minister|politician.*|[Aa]uthorit)',artikel)))
      counts2.append(len(re.findall('[Pp]alest',artikel)))
      
output=zip(filenames, counts1, counts2)
with open("results.csv", mode='w',encoding="utf-8") as fo:
    writer = csv.writer(fo)
    writer.writerows(output)
\end{minted}
\end{frame}



\begin{frame}[fragile,plain]{Example 2: Parsing semi-structured data}
If your data look like this, you can loop over the lines and use regular expressions to extract the info you need!

\begin{lstlisting}
                              All Rights Reserved

                               2 of 200 DOCUMENTS

                                  De Telegraaf

                             21 maart 2014 vrijdag

Brussel bereikt akkoord  aanpak probleembanken;
ECB krijgt meer in melk te brokkelen

SECTION: Finance; Blz. 24
LENGTH: 660 woorden

BRUSSEL   Europa heeft gisteren op de valreep een akkoord bereikt 
over een saneringsfonds voor banken. Daarmee staat de laatste
\end{lstlisting}

\end{frame}



\begin{frame}{Practice yourself!}
	Take some time to write some regular expressions.
	Write a script that
\begin{itemize}
	\item extracts URLS form a list of strings
	\item removes everything that is not a letter or number from a list of strings
\end{itemize}
(first develop it for a single string, then scale up)

More tips:
\huge{\url{http://www.pyregex.com/}}
\end{frame}

