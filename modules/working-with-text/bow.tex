

\subsection{A cleaner BOW representation}

\begin{frame}{Room for improvement}
\begin{description}
	\item[tokenization] How do we (best) split a sentence into tokens (terms, words)?
	\item[pruning] How can we remove unneccessary words?
	\item[lemmatization] How can we make sure that slight variations of the same word are not counted differently?

\end{description}
\end{frame}

\subsubsection{Better tokenization}

\begin{frame}[fragile]{OK, good enough, perfect?}
\begin{block}{.split()}
\begin{itemize}
	\item space $\rightarrow$ new word
	\item no further processing whatsoever
	\item thus, only works well if we do a preprocessing outselves (e.g., remove punctuation)
\end{itemize}
\end{block}
\begin{lstlisting}
docs = ["This is a text",  "I haven't seen John's derring-do. Second sentence!"]
tokens = [d.split() for d in docs]
\end{lstlisting}
\begin{lstlistingoutputtiny}
[['This', 'is', 'a', 'text'], ['I', "haven't", 'seen', "John's", 'derring-do.', 'Second', 'sentence!']]
\end{lstlistingoutputtiny}
\end{frame}


\begin{frame}{OK, good enough, perfect?}
  \begin{block}{Tokenizers from the NLTK pacakge}
    \begin{itemize}
    \item multiple improved tokenizers that can be used instead of .split()
    \item e.g., Treebank tokenizer:
      \begin{itemize}
      \item split standard contractions ("don't")
      \item deals with punctuation
      \item BUT: Assumes lists of \emph{sentences}.
      \end{itemize}
    \item Solution: Build an own (combined) tokenizer (next slide)!
    \end{itemize}
  \end{block}
\end{frame}


\begin{frame}[fragile]{OK, good enough, perfect?}
\begin{minted}{python}
import nltk
import regex

class MyTokenizer:
    def tokenize(self, text):
        tokenizer = nltk.tokenize.TreebankWordTokenizer()
        result = []
        word = r"\p{letter}"
        for sent in nltk.sent_tokenize(text):
            tokens = tokenizer.tokenize(sent)    
            tokens = [t for t in tokens 
                      if regex.search(word, t)]
            result += tokens
        return result
        
mytokenizer = MyTokenizer()
tokens = [mytokenizer.tokenize(d) for d in docs]

\end{minted}

\begin{lstlistingoutputtiny}
[['This', 'is', 'a', 'text'], ['I', 'have', "n't", 'seen', 'John', "'s", 'derring-do', 'Second', 'sentence']]
\end{lstlistingoutputtiny}
\end{frame}









\subsubsection{Stopword removal}



\begin{frame}{Stopword removal}
	\begin{block}{What are stopwords?}
		\begin{itemize}
			\item Very frequent words with little inherent meaning
			\item \texttt{the, a, he, she, \ldots}
			\item context-dependent: if you are interested in gender, \texttt{he} and \texttt{she} are no stopwords. 
			\item Many existing lists as basis
		\end{itemize}
	\end{block}

\end{frame}


\begin{frame}{Stopword removal: What and why?}
	\begin{block}{Why remove stopwords?}
		\begin{itemize}
			\item If we want to identify key terms (e.g., by means of a word count), we are not interested in them
			\item If we want to calculate document similarity, it might be inflated
			\item If we want to make a word co-occurance graph, irrelevant information will dominate the picture
		\end{itemize}
	\end{block}
\end{frame}

\begin{frame}[fragile]{Stopword removal}
	\begin{lstlisting}
from nltk.corpus import stopwords
mystopwords = stopwords.words("english")
mystopwords.extend(["test", "this"])
		
def tokenize_clean(s, stoplist):
    cleantokens = []
    for w in TreebankWordTokenizer().tokenize(s):
        if w.lower() not in stoplist:
            cleantokens.append(w)
	return cleantokens
		
tokens = [tokenize_clean(d, mystopwords) for d in docs]
	\end{lstlisting}
	\begin{lstlistingoutputtiny}
[['text'], ["n't", 'seen', 'John', 'derring-do.', 'Second', 'sentence', '!']]
	\end{lstlistingoutputtiny}

\begin{alertblock}{You can do more!}
	\tiny{For instance, in line 8, you could add an \texttt{or} statement to also exclude punctuation.}
\end{alertblock}

\end{frame}

\begin{frame}[fragile]{Removing punctuation}
	\begin{lstlisting}
from nltk.tokenize import RegexpTokenizer
tokenizer = RegexpTokenizer(r'\w+')
tokenizer.tokenize("Hi all, what's up!")
	\end{lstlisting}  
	\begin{lstlistingoutputtiny}
['Hi', 'all', 'what', 's', 'up']
	\end{lstlistingoutputtiny}
	\begin{lstlisting}
from string import punctuation
doc = "She said, of course i'll come to the party!!!"
"".join([w for w in doc if w not in punctuation])
\end{lstlisting}  
	\begin{lstlistingoutputtiny}
'She said of course ill come to the party'
	\end{lstlistingoutputtiny}
	
\end{frame}

\subsection{Stemming and lemmatization}


\begin{frame}{NLP: What and why?}
	\begin{block}{Why do stemming?}
		\begin{itemize}
			\item Because we do not want to distinguish between smoke, smoked, smoking, \ldots
			\item Typical preprocessing step (like stopword removal)
		\end{itemize}
	\end{block}
\end{frame}

\begin{frame}[fragile]{Stemming and lemmatization}
	\begin{itemize}
		\item Stemming: reduce words to its stem by removing last part (drinking $\rightarrow$ drink)
		\item Lemmatization: find word that you would need to look up in a dictionary (drinking $\rightarrow$ drink, but also went $\rightarrow$ go)
		\item stemming is simpler than lemmatization
		\item lemmatization often better
	\end{itemize}
	\pause
	
	Example below: tokenization and lemmatization with \texttt{spacy} in one go:
	\begin{lstlisting}
		import spacy
		nlp = spacy.load('en')   # potentially you need to install the language model first
		lemmatized_tokens = [[token.lemma_  for token in nlp(doc)] for doc in docs]
	\end{lstlisting}
	\begin{lstlistingoutputtiny}
		[['this', 'be', 'a', 'text'], ['-PRON-', 'have', 'not', 'see', 'John', "'s", 'derring', '-', 'do', '.', 'second', 'sentence', '!']]
	\end{lstlistingoutputtiny}
\end{frame}




\begin{frame}[fragile]{Stemming and stopword removal - let's combine them!}
	\begin{lstlisting}
from nltk.stem.snowball import SnowballStemmer
from nltk.corpus import stopwords
stemmer=SnowballStemmer("english")
mystopwords = stopwords.words("english")
frase="I am running while generously greeting my neighbors"
frasenuevo=""
for palabra in frase.lower().split():
	if palabra not in mystopwords:
	frasenuevo=frasenuevo + stemmer.stem(palabra)  + " "
\end{lstlisting}
Now, {\tt{print(frasenuevo)}} returns:
\begin{lstlisting}
run generous greet neighbor
	\end{lstlisting}
	Perfect!
	\pause
	\small
	Or:
	\begin{lstlisting}
print(" ".join([stemmer.stem(p) for p in frase.lower().split() if p not in mystopwords]))
	\end{lstlisting}
	
\end{frame}

\subsection{ngrams}
\begin{frame}
	Instead of just looking at single words (unigrams), we can also use adjacent words (bigrams).
\end{frame}

\begin{frame}[fragile]{ngrams}
	\begin{lstlisting}
import nltk
texts = ['This is the first text text text first', 'And another text yeah yeah']
texts_bigrams = [["_".join(tup) for tup in nltk.ngrams(t.split(),2)] for t in texts]
print(texts_bigrams)
	\end{lstlisting}
	\texttt{[['This\_is',
		'is\_the',
		'the\_first',
		'first\_text',
		'text\_text',
		'text\_text',
		'text\_first'],
		['And\_another', 'another\_text', 'text\_yeah', 'yeah\_yeah']]
	}
	
	Typically, we would combine both.
	\pause
	\textbf{\textcolor{red}{What do you think? Why is this useful? (and what may be drawbacks?)}}
\end{frame}



\subsection{The order of preprocessing steps}

\begin{frame}{Option 1}
\begin{block}{Preprocessing only through Vectorizer}
``Just use CountVectorizer or Tfidfvectorizer with the appropriate options.''	
\begin{itemize}
	\item pro: No double work, efficient if your main goal is a sparse matrix (for ML?) anyway
	\item con: you cannot ``see'' the preprocessed texts
\end{itemize}
\end{block}
\end{frame}

\begin{frame}[fragile]{Option 2}
	\begin{block}{Extensive preprocessing without Vectorizer}
``Remove stopwords, punctuation etc. and store in a string with spaces''

\begin{lstlisting}
cleaneddocs = [" ".join(re.findall(r"\w\w+", d)).lower() for d in docs]
cleaneddocswithoutstopwords = [" ".join([w for w in d.split() if w not in mystopwords]) for d in cleaneddocs]
\end{lstlisting}
\begin{lstlistingoutputtiny}
['this is text', 'haven seen john derring do second sentence']
['text', 'seen john derring second sentence']	
\end{lstlistingoutputtiny}
{\tiny{Yes, this list comprehension looks scary -- you can make a more elaborate for loop instead}}
	
\begin{itemize}
	\item pro: you can read (and store!) the preprocessed docs
	\item pro: even the most stupid vectorizer (or wordcloud tool) can split the resulting string later on
	\item con: potentially double work (for you \emph{and} the computer)
\end{itemize}
\end{block}
\end{frame}


\question{How would you do it?}

\begin{frame}[plain]
Sometimes, I go for Option 2 because
\begin{itemize}
	\item I like to inspect a sample of the documents
	\item I can re-use the cleaned docs irrespective of the Vectorizer
\end{itemize}

But at other times, I opt of Option 1 instead because
\begin{itemize}
	\item I want to systematically compare the effect of different choices in a machine learning pipeline (then I can simply vary the vectorizer instead of the data)
	\item I want to use techniques that are geared towards little or no preprocessing (deep learning)
\end{itemize}

\end{frame}


\subsection{How further?}


\begin{frame}{Main takeaway}

\begin{itemize}
	\item It matters how you transform your text into numbers (``vectorization'').
	\item Preprocessing matters, be able to make informed choices.
	\item Keep this in mind when we will discuss Machine Learning! It will come back throughout Part II!
\end{itemize}

\begin{itemize}
	\item Once you vectorized your texts, you can do all kinds of calculations (random example: get the cosine similarity between two texts)
\end{itemize}

\end{frame}


\begin{frame}{More NLP}
\begin{description}
	\item[$n$-grams] Consider using $n$-grams instead of unigrams
	\item[collocations]  $n$grams that appear more frequently than expected
	\item[POS-tagging] grammatical function (``part-of-speach'') of tokens
	\item[NER] named entity recognition (persons, organizations, locations)
\end{description}
\end{frame}

\begin{frame}{More NLP}
I \textbf{really} recommend looking into spacy (\url{https://spacy.io}) for advanced natural language processing, such as part-of-speech-tagging and named entity recogntion.
\end{frame}


